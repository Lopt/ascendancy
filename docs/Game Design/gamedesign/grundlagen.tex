\chapter{Grundlagen}
\label{Grundlagen}

\section{Grundprinzipien}
\begin{itemize}
\item Reale Umgebungseinflüsse sollten, soweit es Sinn ergibt, in das Spiel eingebunden werden
\item Das Spiel soll weitgehend auch ohne Bewegung des Spielers in der realen Welt spielbar sein
\item Das bewegen des Spielers in der realen Welt soll stets mehr Vorteile als Stillstehen bieten
\item Die Zahl 6 sollte stets wieder auftauchen. Diese stehen stellvertretend für Feuer, Wasser, Erde, Luft, Gold und Magie.
\item Krieg und Angriffe sollten Zeit- oder Ressourcen-Aufwendiger sein, als die Verteidigung (Faktor 2:1)
\item Ein Schere-Stein-Papier Prinzip soll unbesiegbare Gegner verhindern

\end{itemize}