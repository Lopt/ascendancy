\chapter{Fortführende Ideen}
\label{Fortführende Ideen}

\section{Plündern}
Der Spieler kann bei anderen Hauptgebäuden oder Produktionslinien Rohstoffe stehlen. Unterbunden kann das durch ein Gebäude "Versteck" welches Rohstoffe aufnehmen kann.

Dies fördert das Laufen in der realen Welt, Anfängereb können so weiter fortgeschrittenen Personen mehr Waren stehlen, als diese den Anfängern.

\section{Quests}

\subsection{Capture the Item}
Es wird, innerhalb einer Stunden an einer bestimmten Stelle (ausserhalb jeder Grenzen) ein Gegenstand auftauchen. Dieser Gegenstand muss mit einem Helden zum eigenen Hauptgebäude getragen werden. Dies könnte verschiedene Bonis für den Helden, Rohstoffe oder anderes ergeben.

\subsection{Sacrifice}
Der Spieler muss möglichst viele seiner Einheiten auf einen bestimmten "Opferstein" sterben lassen. Der Spieler der die meisten Opfer gebracht hat, erhält einen Bonus.

\subsection{Race}
Es gibt verschiedene Checkpoints, die in einer bestimmten Reihenfolge von einem Helden durchschritten werden müssen. Der erste, der sämtliche Checkpoints durchquert hat, gewinnt.

\subsection{Zombie Invasion}
Ein NPC Spieler taucht in der Region auf. Zum Beispiel ein Nekromant mit Untoten - tote Gegner verwandeln sich automatisch in Untote. Der, der diesen besiegt, erhält den Bonus.



\section{Ausbau der Helden}
Der Held erhält pro getöteten Gegner Erfahrung und steigt im Level auf. Durch Quests erhält man zudem Gegenstände, welche man dem Helden geben kann. 

\section{Techologiebaum / Zauberbaum?}
Der Spieler kann Zauber erforschen, welche im Kampf eingesetzt werden können. Mitunter auch Gebäude, die erst erforscht werden müssen.
(siehe Master of Magic)

\section{Orb of Might}
Jeder Spieler besitzt einen Orb of Might, der den Spieler, seine Einheiten, Wirtschaft oder Gebäude wesentlich stärkt. Beim zerstören des Hauptgebäudes kann der Orb von einem Helden aufgenommen und zu einem anderen Spieler getragen werden. Der Spieler der den Orb nun besitzt, erhält darauf hin den Bonus.

Der eigene Orbs gibt einen größeren Vorteil als fremde Orbs. Es gibt ein Maximum an Orbs, wie viele ein Spieler haben darf. Die Vorteile werden mit jedem Orb kleiner.

Dies ermöglicht es, Konflikte einzubauen. Die Spieler werden versuchen Orbs zu sammeln, jemand mit vielen Orbs, hat potentiell auch ein großes Risiko, überfallen zu werden. Nicht nur die ehemaligen Besitzer (die sich natürlich zusammenschließen werden, um ihre Orbs zurück zu erhalten), sondern auch andere werden Interesse an den Orbs haben.

\section{Talentbaum}
Der Spieler kann zwischen verschiedenen Talenten wählen.

Dieb: Er kann mehr Rohstoffe von anderen Spielern stehlen.

Farmer: Die Gebäude produzieren mehr Rohstoffe.

Wanderer: Der Spieler erhält mehr Rohstoffe durch seine Umgebung.

Baumeister: Er kann mehr Hauptgebäude / Außenposten bauen.

Nomade: Er kann in verbündeten Kasernen Einheiten bauen, besitzt dafür kein oder nur ein eingeschränktes Hauptgebäude. Falls er keins besitzt, kann er sich in anderen verbündeten Hauptgebäuden "einsiedeln" und erhält durch diese Rohstoffe.

Dies soll dazu dienen, verschiedene Spielstile zu bedienen. Jemand der sich viel bewegt, soll durch ein pures Nomadenleben mehr Einheiten und Rohstoffe erhalten, als jemand der nur sein Zeug baut. Aber: Der Nomade benötigt erst mal einige Verbündete in der Umgebung. Das Nomadenprinzip erlaubt außerdem, bei Überbevölkerung einer Region, mehr aktive Spieler.


\section{Rassen}
Verschiedene Rassen, mit verschiedenen Vor- und Nachteilen.

Zwerge (Erde), Elfen (Luft), Menschen (Feuer, passt das?), ??? (Wasser) 