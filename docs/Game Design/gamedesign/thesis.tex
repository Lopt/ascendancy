% !TeX spellcheck = de-DE
%-------------------------------------------------------------------------------
%   LaTeX Vorlage für Wissenschaftliche Arbeiten an der Fachhochschule Erfurt
%   Version: 0.4
%   Datum:   2011-03-09
%   Autor:   Stephan Rothe
%   email:   stephan_rothe@gmx.de
%----------------------------------------------------------------------------

%-------------------------------------- 
%   Header  
%--------------------------------------      
\documentclass[
		abstractoff,
		appendixprefix=true,		
		a4paper,                  % Papierformat A4
		10pt,                     % Schriftgröße 10pt
		headings=normal,          % kleinere Überschriften verwenden
		chapterprefix=false,      % Einfügen von "Anhang" bzw. "Kapitel" in Überschrift
		oneside,
		openright,                % einseitiges Layout
		titlepage,                % Titleseite verwenden
		listof=totoc,             % alle Listen in das Inhaltsverzeichnis
		headsepline,              % Trennlinie zum Seitenkopf Bereich headings
		plainheadsepline,         % Trennlinie zum Seitenkopf Bereich Plain
		bibliography=totoc,       % das Literaturverzeichnis in den TOC
		parskip=half-,						% Abstand nach Absatz
		numbers=noenddot					% damit hinter der letzten Ziffer kein Punkt steht (Kapitelnummerierung)
	]{scrreprt}                 % verwende KOMA-Report

%--------------------------------------
%   Packages
%--------------------------------------  
\usepackage[utf8]{inputenc}   % Zeichenkodierung
\usepackage[T1]{fontenc}      % Schriftart
\usepackage[ngerman]{babel}   % Babel-System
\usepackage[left=4cm, right=3cm, top=2.5cm, bottom=2.5cm]{geometry} % Seitenränder
\usepackage{array}            % erweiterte Tabelleneigenschaften
\usepackage{graphicx}         % Grafiken
\usepackage{subfigure}        % Grafiken nebeneinander mit (a) und (b)

%---------------------------------------
%  Befehl für Schriftart Helvet / Arial
%---------------------------------------
\renewcommand*{\familydefault}{\sfdefault}
\usepackage[scaled]{helvet}

%%%%%%%%%%%%%%%%2 Alternative Schriftarten%%%%%%%%%%%%%%%%%%%%
%\usepackage{lmodern}					% Schriftart
%\usepackage{palatino}          % Schriftart Palatino (besser am Bildschirm zu lesen)

\usepackage[onehalfspacing]{setspace} % 1,5 facher zeilen bastand aber nur im text nicht in Fußnoten oder verzeichnissen

\usepackage{amssymb}	 				% Mathesymbole
\usepackage{amsfonts}					% mathematische Schriftarten
\usepackage{amsmath}					% Mathepaket
\usepackage{cancel}						% Durchstreichungen wie beim kürzen
\usepackage{mathcomp}					% weitere Symbole
\usepackage{scrhack}
\usepackage[babel,german=quotes]{csquotes}
\usepackage[ngerman]{translator}

%--------------------------------------
%   Metainformation 
%--------------------------------------
\newcommand{\art}{}
\newcommand{\titel}{Game Design}
\newcommand{\untertitel}{}
\newcommand{\autor}{Bernd Schmidt}
\newcommand{\hochschule}{Fachhochschule Erfurt}
\newcommand{\fachgebiet}{Angewandte Informatik}
\newcommand{\keywords}{aa, bb, cc, dd}
\newcommand{\ort}{Erfurt}

	
%-------------------------------------- 
%   PDF Lesezeichen und Hyperlinks
%--------------------------------------
\usepackage[
	pdfauthor={\autor},
	pdftitle={{\titel { - }\untertitel}},
	pdfsubject={{\titel { - }\untertitel}},
	pdfkeywords={\keywords},
	pdfpagelabels = {true},
	pdfstartview = {FitV},
	colorlinks = {true},
	linkcolor = {black},
	citecolor = {black},
	urlcolor = {blue},
	bookmarksopen = {true},
	bookmarksopenlevel = {3},
	bookmarksnumbered = {true},
	plainpages = {false},
	hypertexnames = {false}
]{hyperref}
\usepackage[acronym,toc]{glossaries}


%--------------------------------------
%   Kopf- & Fusszeile 
%--------------------------------------
\usepackage[automark,plainheadsepline,autooneside]{scrpage2}
\pagestyle{scrheadings}
\setheadsepline{.4pt}                                   %Separate Linie im Kopf
\clearscrheadfoot                                       %Kopf und Fuzeile lschen
\ihead[\hochschule]{\hochschule}                        % im Kopf -> links
\ohead[\fachgebiet]{\fachgebiet}                        % im Kopf -> rechts
\cfoot[\pagemark]{\pagemark}                            % Seitenzahl
\renewcommand*{\headfont}{\upshape\sffamily\scriptsize} % Schrift Kopfzeile
\renewcommand*{\footfont}{\normalfont\sffamily\small}   % Schrift Fuzeile


%--------------------------------------
%   Quellcode-Listing Einstellungen
%   http://rasehorn.net/blog/2007/06/latex-code-listings-mit-syntax-highlighting-in-c/
%--------------------------------------

\usepackage{xcolor,listings}                %bindet das Paket Listings ein
\definecolor{comment}{rgb}{.15,.4,.15}     % hellgruen
\definecolor{keywd1}{rgb}{.15,.15,.6}      % dunkelblau
\definecolor{keywd2}{rgb}{.35,.5,.55}      % hellblau
\definecolor{string}{rgb}{.5,.15,.15}      % dunkelrot
\definecolor{gray}{rgb}{0.4,0.4,0.4}
\definecolor{darkblue}{rgb}{0.0,0.0,0.6}
\definecolor{cyan}{rgb}{0.0,0.6,0.6}

% Der lstset-Befehl ermglicht haufenweise Einstellungen zur Formatierung


\lstset{language=java,
basicstyle={\tiny} ,
keywordstyle=\color{blue!80!black!100},
identifierstyle=,
commentstyle=\color{green!50!black!100},
stringstyle=\ttfamily,
breaklines=true,
numbers=left,
numberstyle=\tiny,
frame=single,
backgroundcolor=\color{blue!3},
caption={Java-Code},
captionpos=t,
% %UTF8 gebastle verdammtes tex grml
literate= %
{Ä}{{\"A}}1
v{Ö}{{\"O}}1
{Ü}{{\"U}}1
{ß}{{\ss}}1
{ä}{{\"a}}1
{ö}{{\"o}}1
{ü}{{\"u}}1
{~}{{\textasciitilde}}1
}

\lstdefinelanguage{XML}
{
  morestring=[b]",
  morestring=[s]{>}{<},
  morecomment=[s]{<?}{?>},
  stringstyle=\color{black},
  identifierstyle=\color{darkblue},
  keywordstyle=\color{cyan},
  morekeywords={xmlns,version,type}% list your attributes here
}



%\renewcommand{\lstlistingname}{Quellcodeausschnitt}
%\renewcommand{\lstlistlistingname}{Quellcodeausschnittsverzeichnis}
%\hyphenation{Ver-arbeit-ungs-ketten}
%\hyphenation{ Ein-gangs-pro-zess}
%\hyphenation{Aus-gangs-pro-z#ess}
%\hyphenation{Web-client}
%\hyphenation{DMS-Web-client}
\makeindex 
%-------------------------------------- 
%   Dokumentenbeginn
%--------------------------------------
\begin{document}

%--------------------------------------
%   Titelseite
%--------------------------------------
\begin{titlepage}
	\pdfbookmark[-1]{\titel}{Marke}	%Titel wird bei Lesezeichen angezeigt
	
	\titlehead{
		\begin{flushright}
			\includegraphics[width=0.45\textwidth]{images/Logo_Informatik.pdf}
		\end{flushright}
	}
	\title{\Huge\titel}                          	% Titel der Arbeit
	\author{\textbf{\autor}}                           % Verfasser
	\maketitle                                
\end{titlepage}

%--------------------------------------
%   Seitennummerierung
%--------------------------------------
\pagenumbering{Roman}


%--------------------------------------
%   Inhalts-, Abbildungs-, Tabellenverzeichnis
%--------------------------------------
\pdfbookmark[0]{\contentsname}{tocanc}
\tableofcontents  % Inhalt
%\listoffigures    % Abbildungen
%\listoftables     % Tabellen
%\lstlistoflistings %Listings
\cleardoublepage

%--------------------------------------
%   Zähler für römische Nummerierung
%--------------------------------------
\newcounter{exterior}
\setcounter{exterior}{\value{page}}

\pagenumbering{arabic}



%--------------------------------------
%   Einfügen der Kapitel
%--------------------------------------
\setcounter{page}{1}
\pagenumbering{arabic}
\begin{onehalfspace}
\chapter{Einführung}
\label{Einführung}

Das Spiel kann man als eine Mischung von Battle of Wesnoth, Civilisation und Heroes of Might and Magic sehen. Dabei sollen sämtliche Spieler auf einer gemeinsamen Welt zusammen spielen.

Es soll ein Aufbau Strategiespiel für mobile Plattformen werden, wobei verschiedene reale Umgebungseinflüsse eine wichtige Rolle spielen soll. So soll unter anderem das Wetter Einfluss auf die Kampfstärke haben.

Obwohl es dem Spieler frei stehen soll, sich viel zu bewegen oder an einer Stelle zu bleiben, so sollte Bewegung stets einen Vorteil erhalten.

\chapter{Grundlagen}
\label{Grundlagen}

\section{Grundprinzipien}
\begin{itemize}
\item Reale Umgebungseinflüsse sollten, soweit es Sinn ergibt, in das Spiel eingebunden werden
\item Das Spiel soll weitgehend auch ohne Bewegung des Spielers in der realen Welt spielbar sein
\item Das bewegen des Spielers in der realen Welt soll stets mehr Vorteile als Stillstehen bieten
\item Die Zahl 6 sollte stets wieder auftauchen. Diese stehen stellvertretend für Feuer, Wasser, Erde, Luft, Gold und Magie.
\item Krieg und Angriffe sollten Zeit- oder Ressourcen-Aufwendiger sein, als die Verteidigung (Faktor 2:1)
\item Ein Schere-Stein-Papier Prinzip soll unbesiegbare Gegner verhindern

\end{itemize}
\chapter{Konzept}
\label{Konzept}

\section{Runden und Zeitablauf}
Es existieren keine Spielrunden im klassischen Sinne. Jedes Objekt kann allerdings in einer bestimmten zeitlichen Spanne nur eine Aktion ausführen. Die Aktion wird dabei sofort ausgeführt. Danach benötigt das Objekt eine gewisse Zeit, eh es erneut eine Aktion ausführen darf. Die Zeitspanne ist immer gleich.

Schickt man Einheiten an eine Stelle, so bewegen sie sich umgehend von einem Ort zum anderen. Dort benötigen sie allerdings eine entsprechende Zeit, bevor man sie weiter schicken kann.

Der Zeitabstand sollte dabei klein genug sein, dass der Spieler sich nicht langweilt (oder man findet noch etwas, was er zwischen durch machen kann?), und das genug Zeit besteht, um andere Spieler über die Aktion zu informieren.

\section{Die Welt}
Die Welt besteht aus vielen Hexagonalen Feldern. Das Terrain der Felder ist von dem Terrain der echten Welt abhängig. Das heißt, dort wo in der echten Welt ein See ist, ist in dem Spiel ebenfalls ein See beziehungsweise das Terrain Wasser.

\section{Grenzen}
Der Spieler kann nur innerhalb seiner Reichweite Aktionen ausführen. Einheiten, welche sich außerhalb seiner Reichweite befinden, können nicht bewegt werden. 

Es können verschiedene Gebäude gebaut werden, welche es ermöglichen einen bestimmten Bereich für sich zu beanspruchen. Selbst wenn der Spieler nicht vor Ort ist, kann er innerhalb der Grenzen jede Aktion ausführen. 

So kann der Spieler jederzeit sein Gebiet verteidigen, muss jedoch beim Angriff auf ein fremdes Gebiet selbst vor Ort sein.

Grenzen dürfen sich nicht überlappen. Gebäude, welche Land beanspruchen, können nicht gebaut werden wenn Grenzen sich überlappen würden.




\section{Nebel des Krieges}
Es existiert kein Nebel des Krieges. Es kann alles gesehen werden.

\section{Rohstoffe}

Rohstoffe sind die zentrale Komponente in jedem Aufbau / Strategiespiel. Es existieren 4 Rohstoffe, welche den 4 Elementen entsprechen: Feuer, Wasser, Erde, Luft.

Jeder Rohstoff soll sich bei der Erstellung von Einheiten und Gebäuden entsprechend ihres Aspektes auswirken. Baut man eine Mauer mit Erde, hält sie mehr aus. Baut man sie mit Feuer, so schädigt sie Angreifer zusätzlich.

Der Rohstofferhalt ist von der realen Umgebung abhängig. Dort wo Wasser ist, erhält man eben Wasser. Der Spieler erhält Rohstoffe abhängig von dem Terrain welches sich innerhalb seines Standpunkts befindet, in regelmäßigen Zeitabständen.

Zusätzlich kann er durch Gebäude, wie Steinbrüche, Windmühlen... Rohstoffe erhalten.

\begin{itemize}
\item Gold (Überall)
\item Feuer (Wald, Park, ...)
\item Wasser (Seen, Flüsse, Meer, ...)
\item Erde (Straßen, Städte, Dörfer, Gebäude, ...)
\item Luft (Acker, Wiesen, Alles andere...)
\item Magie (-)
\end{itemize}

\section{Einheiten}

Einheiten sind auf der Karte sichtbare Objekte. Sie können sich nur an Land bewegen. Sie können mit wenig Gold erschaffen werden.
\begin{itemize}
\item Angriff 
\item Verteidigung
\item Leben
\item Zugweite / Aktionen
\end{itemize}


\subsection{Nahkämpfer}
Die geläufigsten Einheiten sollen die Nahkämpfer sein. Sie sollen recht billig sein.


\begin{itemize}
\item Krieger (Nichts)
\item Schwertkämpfer (Wasser: Leben)
\item Speerkämpfer (Luft: Schnell)
\item Schwergerüstet (Erde: Verteidigung)
\item Barbar (Feuer: Angriff)
\item Kampfmagier (Alles)
\end{itemize}

\subsection{Magier}
Magier können nur mit Magie erstellt werden. Ihre Stärke liegt in einer Spezialfähigkeit, abhängig von ihrem Element. 

\begin{itemize}
\item Gold (???)
\item Feuer (???)
\item Wasser (Heilen)
\item Luft (???)
\item Stein (???)
\item Magie (??)
 Ideen: Kurzfristige Portale, Über Wasser laufen, Unsichtbarkeit, Illusion, Kurzfristige Mauer, Einheiten bekehren, Schilde, ...
\end{itemize}

\subsection{Held}
Helden sollen entweder extrem selten oder einmalig sein. Man erhält diese nur schwer.

Sie sind herausragende Kämpfer und jeder anderen Einheit überlegen. Zusätzlich erhalten sämtliche Verbündeten in der Nähe einen Multiplikator, abhängig von dem Element des Helden.

\begin{itemize}
\item Gold (???)
\item Feuer (Angriffsmultiplikator)
\item Erde (Verteidigungsmultiplikator)
\item Wasser (Lebensmultiplikator)
\item Luft (Geschwindigkeitsmultiplikator)
\item Magie (???)
\end{itemize}

\subsection{Fernkämpfer und Belagerungsmaschinen}
Fernkämpfer wurden aus Balancing Gründen nicht mit hinein genommen. Falls jemand verspätet oder gar nicht reagiert, wäre es mit Fernkämpfern möglich, große Teile der feindlichen Armee ohne eigene Verluste zu überwältigen. Dies ist nicht gewollt. 

Möglicherweise lässt sich ein Balancing erreichen, in dem die Fernkämpfer sehr lange Nachladezeiten besitzen. Die Nachteile müssten nur so groß sein, das es nicht praktikabel ist, mit Fernkämpfern eine ganze Armee zu zerstören.

Belagerungsmaschinen, ein Trébuchet, welches relativ selten schiesst, allerdings eine hohe Reichweite und hohen Schaden besitzt, wäre denkbar. Womöglich mit zusätzlichen Schaden gegen Gebäude, dafür verringerten Schaden gegenüber Einheiten.


\subsection{Schiffe}
Schiffe oder andere Wassertaugliche Fahrzeuge existieren nicht. Der Spieler muss in der Nähe sein, um Einheiten bewegen zu können, auf Gewässern würden aber ohnehin sehr wenige Spieler unterwegs sein.

Es ist zudem ein Taktisches Element, dies gibt den Spielern die Möglichkeit, ihre Verteidigung an Flüssen oder Gewässern auszurichten.



\section{Gebäude}
Gebäude sind auf der Karte sichtbare, stationäre Objekte. Sie können nicht auf Wasser gebaut werden.
\begin{itemize}
\item Angriff 
\item Verteidigung
\item Leben
\end{itemize}

Die Größe sowie Form von den Gebäuden kann dabei variieren.

Gebäude können nur innerhalb der eigenen Grenzen gebaut werden. Ausnahme hierbei sind jene Gebäude, welche Land beanspruchen, wie Hauptgebäude und Außenposten.


\section{Hauptgebäude}
Das Hauptgebäude ist die Zentrale des eigenen Landes. Das Hauptgebäude beansprucht das Land in der näheren Umgebung. 

Die Zerstörung eines Hauptgebäudes sollte möglichst langwierig sein, um dem Verteidigten Spieler selbst in Nacht- und Nebelaktionen noch die Chance zu geben, 

Im Gegensatz dazu gibt es noch Außenposten, jene besitzen eine geringere Grenze. Falls ein Hauptgebäude zerstört wird, kann ein Außenposten als neues Hauptgebäude ausgewählt werden.

Personen ohne Außenposten können sich einen beliebigen, freien Platz für ein neues Hauptgebäude aussuchen.

Hauptgebäuden und Außenposten können benannt werden.

\section{Portale}
Portale sind eine Möglichkeit, um Einheiten schnell von einem Punkt zu einem anderen zu bringen. Sie dienen dabei als zeitverzögerter Langstreckentransporter, die es auch ermöglichen, Einheiten von einer realen Stadt zu einer anderen zu transportieren.

\section{Wohngebäude}
Wohngebäude erhöhen das Bevölkerungslimit. Wenn man über dem Limit ist, können keine neuen Einheiten mehr erschaffen werden.

\section{Steinbruch, Windmühlen, Brunnen, ???}
Dient zum Rohstofferhalt innerhalb eines Bereich des jeweiligen Gebäudes. Brunnen sollten, da man nicht auf Wasser bauen können soll, eine höhere Reichweite haben.

\section{Türme}
Türme bringen Multiplikatoren für Angriff (Feuer), Verteidigung (Erde), Leben (Wasser) und Signalisieren  dem Spieler über Einheitenbewegungen (Luft). 


\section{Mauer}
Mauern sollen feindliche Truppen das Einmaschieren erschweren. Es soll dabei verschiedene Mauern geben:

\begin{itemize}
\item Feuer (Brennt, Macht Schaden)
\item Erde (Hält mehr aus)
\item Luft (??? Buffs / Debuffs, Multiplikatoren))
\item Wasser (??? Buffs / Debuffs, Multiplikatoren)
\end{itemize}

\section{Einbeziehen der realen Welt}
Das Spiel soll darauf ausgelegt werden, bestimmte Faktoren aus der realen Welt einzubeziehen. Dabei wird zwischen globalen, lokalen und benutzerspezifischen Faktoren unterschieden. 

\subsection{Globale Faktoren}
\begin{itemize}
\item Realer Goldpreis (Tauschpreise im Spiel)
\item Kohle
\item Holzwert-Preise
\item DAX
\end{itemize}

\subsection{Lokale Faktoren}
\begin{itemize}
\item Wetter 
\item Terrain 
\item Uhrzeit (time.h)
\item Jahreszeit (time.h)
\item Arbeitslosenrate in der Gegend
\item Grundstückpreise
\item Verbrechensrate
\item Geburt / Sterberate
\end{itemize}

\subsection{Benutzerspezifische Faktoren}
\begin{itemize}
\item Betriebssystem des verwendeten Mobiltelefons
\item Spielzeiten
\end{itemize}


\section{Kampfsystem}
\begin{itemize}
\item Jede Einheit kann in einem definierten Zeitraum einmalig angreifen
\item Die Verteidigende Einheit schlägt zurück, dies zählt nicht als Angriff
\item Der Verteidiger erhält Vorteile, der Angreifer einen Malus (?)
\item Es existieren Zufallskomponenten
\end{itemize}


Wasser ist gut gegen Feuer. Feuer ist gut gegen Erde. Erde ist gut gegen Luft. Luft ist gut gegen Wasser. 


\chapter{Fortführende Ideen}
\label{Fortführende Ideen}

\section{Plündern}
Der Spieler kann bei anderen Hauptgebäuden oder Produktionslinien Rohstoffe stehlen. Unterbunden kann das durch ein Gebäude "Versteck" welches Rohstoffe aufnehmen kann.

Dies fördert das Laufen in der realen Welt, Anfängereb können so weiter fortgeschrittenen Personen mehr Waren stehlen, als diese den Anfängern.

\section{Quests}

\subsection{Capture the Item}
Es wird, innerhalb einer Stunden an einer bestimmten Stelle (ausserhalb jeder Grenzen) ein Gegenstand auftauchen. Dieser Gegenstand muss mit einem Helden zum eigenen Hauptgebäude getragen werden. Dies könnte verschiedene Bonis für den Helden, Rohstoffe oder anderes ergeben.

\subsection{Sacrifice}
Der Spieler muss möglichst viele seiner Einheiten auf einen bestimmten "Opferstein" sterben lassen. Der Spieler der die meisten Opfer gebracht hat, erhält einen Bonus.

\subsection{Race}
Es gibt verschiedene Checkpoints, die in einer bestimmten Reihenfolge von einem Helden durchschritten werden müssen. Der erste, der sämtliche Checkpoints durchquert hat, gewinnt.

\subsection{Zombie Invasion}
Ein NPC Spieler taucht in der Region auf. Zum Beispiel ein Nekromant mit Untoten - tote Gegner verwandeln sich automatisch in Untote. Der, der diesen besiegt, erhält den Bonus.



\section{Ausbau der Helden}
Der Held erhält pro getöteten Gegner Erfahrung und steigt im Level auf. Durch Quests erhält man zudem Gegenstände, welche man dem Helden geben kann. 

\section{Techologiebaum / Zauberbaum?}
Der Spieler kann Zauber erforschen, welche im Kampf eingesetzt werden können. Mitunter auch Gebäude, die erst erforscht werden müssen.
(siehe Master of Magic)

\section{Orb of Might}
Jeder Spieler besitzt einen Orb of Might, der den Spieler, seine Einheiten, Wirtschaft oder Gebäude wesentlich stärkt. Beim zerstören des Hauptgebäudes kann der Orb von einem Helden aufgenommen und zu einem anderen Spieler getragen werden. Der Spieler der den Orb nun besitzt, erhält darauf hin den Bonus.

Der eigene Orbs gibt einen größeren Vorteil als fremde Orbs. Es gibt ein Maximum an Orbs, wie viele ein Spieler haben darf. Die Vorteile werden mit jedem Orb kleiner.

Dies ermöglicht es, Konflikte einzubauen. Die Spieler werden versuchen Orbs zu sammeln, jemand mit vielen Orbs, hat potentiell auch ein großes Risiko, überfallen zu werden. Nicht nur die ehemaligen Besitzer (die sich natürlich zusammenschließen werden, um ihre Orbs zurück zu erhalten), sondern auch andere werden Interesse an den Orbs haben.

\section{Talentbaum}
Der Spieler kann zwischen verschiedenen Talenten wählen.

Dieb: Er kann mehr Rohstoffe von anderen Spielern stehlen.

Farmer: Die Gebäude produzieren mehr Rohstoffe.

Wanderer: Der Spieler erhält mehr Rohstoffe durch seine Umgebung.

Baumeister: Er kann mehr Hauptgebäude / Außenposten bauen.

Nomade: Er kann in verbündeten Kasernen Einheiten bauen, besitzt dafür kein oder nur ein eingeschränktes Hauptgebäude. Falls er keins besitzt, kann er sich in anderen verbündeten Hauptgebäuden "einsiedeln" und erhält durch diese Rohstoffe.

Dies soll dazu dienen, verschiedene Spielstile zu bedienen. Jemand der sich viel bewegt, soll durch ein pures Nomadenleben mehr Einheiten und Rohstoffe erhalten, als jemand der nur sein Zeug baut. Aber: Der Nomade benötigt erst mal einige Verbündete in der Umgebung. Das Nomadenprinzip erlaubt außerdem, bei Überbevölkerung einer Region, mehr aktive Spieler.


\section{Rassen}
Verschiedene Rassen, mit verschiedenen Vor- und Nachteilen.

Zwerge (Erde), Elfen (Luft), Menschen (Feuer, passt das?), ??? (Wasser) 
\chapter{Zusammenfassung}
\label{Zusammenfassung}

\section{Fazit}
\label{Fazit}


\end{onehalfspace}



%--------------------------------------
%   Seitennummerierung - Römisch
%--------------------------------------
\pagenumbering{Roman}
\setcounter{page}{\value{exterior}}

%--------------------------------------
%   Literaturverzeichnis mit BiBTeX
%--------------------------------------
\nocite{*} % baue alle Eintrge aus bib/literatur ein
\bibliographystyle{alphadin}

\bibliography{literatur}
\clearpage
\glsaddall
%--------------------------------------
%   Anhang
%--------------------------------------

%--------------------------------------
%   Dokumentenende
%--------------------------------------
\end{document}


