Alle Angaben ohne Gewähr. Die Einrichtung wurde bisher nur auf wenigen Machinen vorgenommen.

\section*{Einrichtung\-:}

\subsection*{Allgemein}

Es muss das Git Repository heruntergeladen werden. ascendancy.\-git runterladen, stable auschecken und, falls Serverbetrieb, die Weltdaten in server/data herunterladen.

git clone ssh\-://username.no-\/ip.\-biz/home/git/ascendancy.git cd ascendancy checkout stable cd server/data git clone ssh\-://username.no-\/ip.\-biz/home/git/ascendancy-\/world.git

\subsection*{Windows (Android \& \hyperlink{namespaceServer}{Server})}

Benötigt\-:
\begin{DoxyItemize}
\item Installiertes V\-S2012 (für \hyperlink{namespaceServer}{Server} und \hyperlink{namespaceClient}{Client}) oder höher (nur \hyperlink{namespaceClient}{Client} -\/ \hyperlink{namespaceServer}{Server} benötigt V\-S2012) \href{https://www.visualstudio.com/downloads/download-visual-studio-vs}{\tt https\-://www.\-visualstudio.\-com/downloads/download-\/visual-\/studio-\/vs}
\item Xamarin Studio (Empfohlen) oder Xamarin für Visual Studio -\/ minimal Probe-\/ oder Indie-\/\-Zugang \href{http://xamarin.com/studio}{\tt http\-://xamarin.\-com/studio}
\item Beliebiger Android Emulator O\-D\-E\-R Genymotion (Empfehlung) O\-D\-E\-R echtes Gerät \href{https://www.genymotion.com/}{\tt https\-://www.\-genymotion.\-com/}
\end{DoxyItemize}

Es muss das Git Repository heruntergeladen werden. Die Git-\/\-Befehle für Linux gelten auch hier\-: ascendancy.\-git runterladen, stable auschecken und, falls Serverbetrieb, die Weltdaten in server/data herunterladen.

\subsection*{Mac O\-S (i\-O\-S \& Android \& \hyperlink{namespaceServer}{Server})}

Benötigt\-:
\begin{DoxyItemize}
\item Xamarin Studio (Empfohlen) oder Xamarin für Visual Studio -\/ minimal Probe-\/ oder Indie-\/\-Zugang \href{http://xamarin.com/studio}{\tt http\-://xamarin.\-com/studio}
\end{DoxyItemize}

(T\-O\-D\-O\-: Chris bitte hinzufügen)

\subsection*{Linux (\hyperlink{namespaceServer}{Server})}

Offizielles Mono Repo hinzufügen\-:

sudo apt-\/key adv --keyserver hkp\-://keyserver.ubuntu.\-com\-:80 --recv-\/keys 3\-F\-A7\-E0328081\-B\-F\-F6\-A14\-D\-A29\-A\-A6\-A19\-B38\-D3\-D831\-E\-F echo \char`\"{}deb http\-://download.\-mono-\/project.\-com/repo/debian wheezy main\char`\"{} $\vert$ sudo tee /etc/apt/sources.list.\-d/mono-\/xamarin.list sudo apt-\/get update

Mono und Abhängigkeiten isntallieren (Mono, I\-D\-E, Webserver und P\-C\-L für die Core-\/\-Bibliothek)\-:

sudo apt-\/get install mono monodevelop mono-\/xsp4 referenceassemblies-\/pcl

\section*{Eigenen \hyperlink{namespaceServer}{Server} aufsetzen}

Der \hyperlink{namespaceServer}{Server} kann nun mit einer beliebigen Xamarin, Mono\-Develop Version oder Visual\-Studio 2012 kompiliert und ausgeführt werden.

Damit der \hyperlink{namespaceServer}{Server} von einem \hyperlink{namespaceClient}{Client} genutzt wird, muss der \hyperlink{namespaceServer}{Server} von aussen erreichbar sein. Es muss gegebenfalls eine Port Weiterleitung eingerichtet werden. Im \hyperlink{namespaceClient}{Client} unter Client\-Constants muss die \hyperlink{namespaceServer}{Server} Adresse angepasst werden. (in Client\-Constants.\-cs die Konstanten D\-A\-T\-A\-\_\-\-S\-E\-R\-V\-E\-R = Apache/\-Datenserver; L\-O\-G\-I\-C\-\_\-\-S\-E\-R\-V\-E\-R = Spieleserver ändern)

\section*{Probleme\-:}


\begin{DoxyItemize}
\item Falls keine Verbindung zum \hyperlink{namespaceServer}{Server} existiert, läd er nur den Startbildschirm -\/$>$ Netzwerk einrichten, im Smartphone-\/\-Browser testen und App neustarten Es kann kontrolliert werden, ob der \hyperlink{namespaceServer}{Server} läuft, in dem folgende Adressen im Webbrowser gestartet werden\-: \href{http://derfalke.no-ip.biz/}{\tt http\-://derfalke.\-no-\/ip.\-biz/} \href{http://derfalke.no-ip.biz:9000}{\tt http\-://derfalke.\-no-\/ip.\-biz\-:9000}
\item Falls kein G\-P\-S existiert, sieht man nur grüne Wiese -\/$>$ G\-P\-S ändern. Gültige Koordinaten\-: Erfurt und Umgebung
\end{DoxyItemize}

\section*{Aufbau}

Das Projekt Ascendancy besteht aus 3 kompilierbaren Projekten (server, client, test) und einer Bibliothek (core). Das test-\/\-Projekt diente zum ersten testen der Funktionalität, kann im aktuellen Stand aber ignoriert werden, da der \hyperlink{namespaceClient}{Client} mittlerweile diese Funktionalität nutzt.

\subsection*{\hyperlink{namespaceCore}{Core}}

Der \hyperlink{namespaceCore}{Core} beinhaltet sämtliche Kernfunktionalität, welche sich \hyperlink{namespaceServer}{Server} und \hyperlink{namespaceClient}{Client} teilen (sollten)\-:
\begin{DoxyItemize}
\item Spielelemente (Entititäten, Regionen, Definitionen)
\item Verwalten der Spielelementen (Manager)
\item Positionenstransformationen der verschiedenen Koordinatensysteme (Lat\-Lon, Spielfeld-\/\-Position, Region+\-Cell)
\item Die komplette Spielelogik (In Form von Actions, plus alle zusätzlichen Algorithmen wie A$\ast$ zur Wegfindung)
\item Klassen für den Datentransfer
\end{DoxyItemize}

\subsection*{\hyperlink{namespaceClient}{Client}}


\begin{DoxyItemize}
\item G\-P\-S
\item Anzeige des Logos
\item Laden von Definitionen vom \hyperlink{namespaceServer}{Server}
\item Login beim \hyperlink{namespaceServer}{Server}
\item Anzeige des Spielfelds anhand der vom \hyperlink{namespaceServer}{Server} geladenen Spielinformationen
\item Laden der Entities und Aktionen vom \hyperlink{namespaceServer}{Server} (sowie Anzeige)
\item Rudimentäre Implementation von Aktions-\/\-Animationen (Bewegungen)
\end{DoxyItemize}

\subsection*{\hyperlink{namespaceServer}{Server}}


\begin{DoxyItemize}
\item Verwaltung und Authentifizierung von Spielern (registrierung, sessions, einloggen)
\item Verwaltung welche Aktionen ausgeführt werden können
\item Verwaltung welche Spieler welche Informationen erhalten (Ortsabhängig, und abhängig welche der Spieler diese bereits erhielt))
\item Vermeidung von Threading-\/\-Problemen bei mehreren Spielern (siehe ascendancy/docs/\-S\-K\-B\-S\-\_\-\-Bernd\-\_\-\-Schmidt.\-pdf)
\item Speicherung von persistenten Zuständen (Datenbank, implementiert aber ungenutzt)
\item Scripts zum Konvertieren von Open\-Street\-Maps-\/\-Daten in nutzbare J\-S\-O\-N Dateien
\end{DoxyItemize}

\section*{Rechtliches, Copyright und Lizenzrechtliches}

siehe ascendancy/licences
\begin{DoxyItemize}
\item Kartendaten von Open\-Street\-Maps ( Contributors)
\item Bilder von Battle of Wesnoth
\item Bilder von pixel32 
\end{DoxyItemize}